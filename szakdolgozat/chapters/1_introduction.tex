\Chapter{Bevezetés}

A dolgozat fő célja egy Java asztali alkalmazás elkészítése volt, ami képes többgépes környezetben is biztonságosan eltárolni a felhasználó adatait és visszaadni azokat szinkronizált módon az eszközök között.
\\ \indent A program készítése során különböző problémákat kellett megoldanom, mint például a megfelelő titkosítási algoritmus kiválasztása, adatszinkronizáció elérése különböző eszközök között, esztétikus Java grafikus felhasználói felület létrehozása.
\\ \indent A dokumentum főleg a felsorolt problémakörök bemutatásáról szól. 
\\ Megpróbálja részletesen ismertetni adattitkosítás során előfordulható problémákat. 
\\ Jellemzi, összeveti ezen problémák megoldási lehetőségeit. Ezek közé értendő az adatszerkezetek és adattárolási módok, adatbázis típusok, adatbázis titkosítási módszerek, titkosítási algoritmusok jellemzése. 
\\ Publikált szoftverek is bemutatásra kerülnek, amelyek valamilyen szinten érintik bármely problémakört, valamint az alkalmazott megoldási módszereik, amennyire a hivatalos dokumentációkban ezek elérhetőek voltak.
