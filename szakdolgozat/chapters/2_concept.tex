\Chapter{Koncepció}

\Section{Adattárolás}
Az adatok tárolása megoldható adatbázisok segítségével és fájlokba.

\subsection{XML és JSON}
 Ilyen adatátviteli formátum közül a leggyakrabban használt az XML és a JSON. Céljuk, hogy platform-független, programozási nyelv független, alkalmazás független adatleíró formátumot adjanak, ami képes adatok nagy mértékű ávitelére és feldolgozására.
\newline
\\ \underline{XML – eXtensible Markup Language}
\vspace{10pt}
\newline \noindent Rugalmas, könnyen értelmezhető emberek és számítógép számára is, HTML-hez hasonló szintaxisú nyelv. 
\vspace{5pt}\\Felépítése tag-ekből áll. Nem előre definiáltak a tag-ek, ez azt jelenti, hogy mi hozzuk létre őket. Egy példa tag: <autó>…. </autó>.
\vspace{5pt}\\Első a nyitó tag, a második ferde vonallal jelölt tag a záró tag.
\\Két tag között lehet általunk megadott szöveg (pl. autó márka) vagy további tag-ek, gyerekelemek (pl rendszám, szín). Tag megadásának másik verziója: <autó/>
\vspace{5pt}\\Nagy mennyiségű tag használata esetén robosztus méreteket is elérhet.
\newline
\\ \underline{JSON – JavaScript Object Notation}
\vspace{10pt}
\newline \noindent Rugalmas, könnyen értelmezhető emberek és számítógép számára is
\\ Adatok név-érték párokban szerepelnek (pl. ”keresztnév” : ”Dániel”).
\\ Adatok vesszővel vannak elválasztva (pl. ”márka” : ”Mercedes”, ”szín” : ”piros”)
\\ \{ \} – objektumokat tartalmaz (pl. {”keresztnév” : ”Dániel”, ”vezetéknév” : ”Nemcsik”})
\newline [ ] –tömböt tartalmaz (pl. \{ ”diákok”:[
\\  \indent	\{”keresztnév” : ”Dániel”, ”vezetéknév” : ”Nemcsik”\},
\\  \indent \{”keresztnév” : ”Tibor”, ”vezetéknév” : ”Nagy”\}
]\}

\begin{center}
	

	\begin{tabular}{|p{7.2cm}|p{7.2cm}|}
		\hline
		\textbf{JSON} & \textbf{XML} \\
		\hline
		JavaScript nyelven alapszik & SGML szabványból származtatott \\
		\hline
		Tábla(kulcs-érték) felépítés & Fa felépítés \\
		\hline
		Könnyedén értelmezhető & Tag felépítés miatt, aki nem tudja hogy kell értelmezni, annak nehezebb dolga lesz mint egy JSON file-lal lenne\\
		\hline
		Nem támogat kommenteket & Támogat kommenteket \\
		\hline
		Adatok szerver és böngésző oldal közötti hordozására preferált & Szerver oldali információ tárolásra preferált\\
		\hline
		Kevésbé terjedelmes és gyorsabb & Több szó szerepel benne, így nagyobb terjedelmű. Lassabb.\\
		\hline
		Kisebb file méret & Nagyobb file méret\\
		\hline
		String, számok, tömbök, objektumok és boolean értékeket támogat & Komplexebb adattípusokat is támogat például képek, nem-primitív adattípusok, stb..\\
		\hline
		Minden böngésző támogatja & Legtöbb böngésző támogatja\\
		\hline
		Adatcsere formátumtípusú & Jelölőnyelv formátumtípusú\\
		\hline
		Adat megjelenítésére nem alkalmas & Mivel jelölőnyelv (markup language), ezért adat megjelenítésére is alkalmas\\
		\hline
		Egy szabványos javascript függvénnyel használat előtt elemezni/felbontani/szeparálni (parse) kell a file-t. & Adott programozási nyelvnek megfelelően kell elemezni/felbontani/szeparálni (parse) a file-t.\\
		\hline
		Könnyedén elemezhető, kevés kód szükséges hozzá & Nehezebben elemezhető a tartalma\\
		\hline
		Adat-orientáltnak nevezik & Dokumentum-orientáltnak nevezik\\
		\hline
	\end{tabular}
\end{center}

\vspace{6pt}
Tegyük fel, hogy egy alkalmazás adatai vagy JSON vagy XML típusú fájlként tárolunk lokálisan, a felhasználó eszközén. \textbf{Lokális fájl} alapú tárolást az alábbi tulajdonságokkal jellemezhetjük:
\vspace{5pt}\\- Gyors és egyszerű hozzáférés az adatokhoz lokális tárolásból kifolyólag.
\vspace{5pt}\\- Könnyedén implementálható megoldás, évtizedek óta alkalmazott módszer az adatok lokális fájlként való tárolása.
\vspace{5pt}\\- Könnyedén átmásolható fizikai adathordozóra, így potenciálisan növelve a védelmet.
\vspace{5pt}\\- Adathordozó elhagyása / lopása, vagy fájlok véletlen törlése biztonsági mentés hiánya esetén adatvesztést eredményezhet.
\vspace{5pt}\\- Több hasonló fájl esetén gyakori jelenség az adatduplikáció. 


\newpage
\subsection{Adatbázis}

Adatbázisok kapcsán minden bizonnyal hallottunk már az SQL-ről, legtöbb relációs adatbázis ezt a nyelvet használja, egy struktúrált lekérdező nyelv. Feltételezem az SQL ismeretét, ezért bővebb jellemzésére nem térek ki. 
Adatbázisoknak viszont lehetnek különböző típusúak, különböző adatmodellűek. Az említett relációs adatbázis (RDBMS) az egyik, a másik amit bemutatok pedig a NoSQL. 

\begin{center}
	
	
	\begin{tabular}{|p{7.2cm}|p{7.2cm}|}
		\hline
		\textbf{RDBMS} & \textbf{NoSQL} \\
		\hline
		Adatok összefüggése táblák közötti kapcsolattal határozható meg, struktúrált formátumú tárolási mód. & Dinamikus séma, nem struktúrált adatnak nevezzük. Nem relációs adatmodell. Adatok gyakran kulcs-érték párokban szerepelnek vagy JSON formátumban.\\
		\hline
		Felépítéséből adódóan adatmodell változásokat nehezebb implementálni. & Könnyebb implementálni adatmodell változásokat, mivel az adatok között nincs feltétlenül kapcsolat. \\
		\hline
		Fix, meghatározott adat felépítés esetén a legjobb használni. & Nem struktúrált, komplex adatok esetén gyorsabb, kevesebb erőforrás igényű lehet, mint egy RDBMS.\\
		\hline
	\end{tabular}
\end{center}

\vspace{10pt}
\noindent Azonfelül, hogy egy adatbázis milyen adatmodellű, lehet lokális, a felhasználó eszközén tárolandó adatbázis és távoli, felhőalapú, cloud adatbázis.
\\ Térjünk ki a lokális és felhőalapú adatbázisok közötti összehasonlításra: \newline


\noindent \textbf{Lokális adatbázis}, hasonló \underline{előnyei} vannak, mint egy helyileg tárolt fájlnak:
\\- Egyszerű hozzáférés az adatokhoz, teljes adat kontrollálás.
\vspace{5pt}\\- Nincs szükség internetkapcsolatra.
\vspace{5pt}\\- Gyorsabb, mint egy távoli adatbázis, mivel gyorsabb a helyi lemez elérése, mint egy távoli esetében a hálózaton keresztüli kommunikáció.
\vspace{5pt}\\ \underline{Hátrány:}
\\- Nehéz az adatmegosztás külső résztvevővel (másik számítógép lokális adatbázisa).
\vspace{5pt}\\- Ha az eszköz olyan állapotba kerül, hogy a fájlokhoz nem lehet hozzáférni, akkor elveszlik minden adat.\newline

\noindent \textbf{Felhőalapú adatbázis}:
\\- Bárhonnan elérhető, internet elérés szükséges. Negatívuma ugyanebből származik, ha nincs internet, nem lehet elérni.
\vspace{5pt}\\- Adatok nem helyileg a számítógépen tárolódnak, emiatt lassabb lehet a hozzáférés.Figyelembe kell venni, hogy más szerverek / alkalmazások is használhatják ugyanazt a hálózatot, ami szintén befolyásolhatja az adatok elérését.
\vspace{5pt}\\- Teknikai hibák ellenére (számítógép elromlik) az információ biztonságba marad a távoli adatbázison.
\vspace{5pt}\\- Később lesz szó a szinkronizálásról, de megemlítem itt, hogy egy cloud adatbázis esetén a szinkronizáció bonyolult megoldása kikerülhető.

\newpage
\Section{Adat állapot}
Adat állapota alatt két lehetséges állapotra gondolok.

\subsection{Data-at-rest}
Nyugvó adatnak olyan adatokat nevezünk, amelyek nem mozognak eszközről eszközre, vagy hálózatról hálózatra. Általában merevlemezen vagy pendrive-on tárolódnak.
\vspace{5pt}\\A nyugalmi adatvédelem célja a bármilyen eszközön vagy hálózaton tárolt inaktív adatok védelme.
\vspace{5pt}\\A nyugvó adatokat általában kevésbé sebezhetőnek tartják, mint a mozgásban lévő adatokat (data-in-motion), gyakran értékesebbnek is találják ezek tartalmát. Nyugvó adatok esetében az ellopható információ mennyisége sokkal nagyobb lehet mint az éppen úton levő adatoké.
\vspace{5pt}\\Adatok biztonsága a szükséges óvintézkedések megtételétől függ.
\vspace{5pt}\\Egy cég legtöbb esetben bizalmas adatit saját hálózatán belül tárolja, azok mégis veszélyben lehetnek rosszindulatú külső és belső fenyegetések miatt. Egy betolakodó könnyedén elérheti egy cég adatait, ha sikerül jogtalanul hozzáférnie egy számítógépükhöz vagy egy lopott eszközt feltörnie.
Data-at-rest típusú adatok védelme érdekében az egyik legjobb és legegyszerűbb módszer ezek titkosítása.


\subsection{Data-in-motion / data-in-transit}
Mozgásban lévő adatnak vagy tranzitadatnak olyan adatokat nevezünk, amelyek aktívan mozgásban vannak egyik helyről a másikra vagy az interneten vagy magánhálózaton keresztül.
\vspace{5pt}\\Mivel az adatok mozgásban vannak, ezért kevésbé biztonságosnak tekinthetők. Célja olyan adatok védelme amelyek például belső hálózaton belül mozognak, vagy helyi tárolóeszközről felhőtípusú tárolóeszközre.
\vspace{5pt}\\Tranzitadat esetében is egy kiváló biztonsági intézkedés az adatok titkosítása. Védi az adatokat, ha két fél közötti kommunikációt ’lehallgatják’.
Ez a védelem az adatok titkosításával biztosítható, még mozgatás előtt. Ez lehet talán a legfontosabb része a mozgásban lévő adatok védelmének, illetve a megfelelő kulcskezelés. Végpontok hitelesítése és adat érkezésekor való visszafejtése és ellenőrzése is tovább fokozhatja a védelmet. 

\newpage
\Section{Adattitkosítási módszerek}

Ezt a szegmenst két részre fogom bontani. Először is az adatbázis titkosítási módszereket fogom bemutatni, ezek olyan titkosítási eljárások, amik kizárólag adatbázis titkosítására lettek kifejlesztve. A másik csoportba minden más olyan titkosítási módszer tartozik, amik nem adatbázishoz köthetők.

\subsection{Adatbázis titkosítási módszerek} 

Egy olyan folyamatot nevezhetünk adatbázis titkosításnak, ami egy algoritmus segítségével az adatbázisban tárolt adatokat titkosított szöveggé (cipher text) alakítja, ami értelmezhetetlen a megfelelő kulcs ismerete nélkül.
\\Célja, hogy megvédje az adatainkat a potenciális fenyegetések ellen. Hogyha egy hacker valahogy sikeresen feltöri az adatbázist, akkor számára értéktelen, értelmezhetetlen szöveggel fog találkozni.
\\Többféle titkosítási technika is létezik, melyek közül a legleterjedtebbek:\newline

\noindent\textbf{Transparent Data Encryption (TDE) (Átlátható adat titkosítás):}\newline
\noindent - Teljes adatbázist, úgynevezett ’nyugvó adatokat’ (data-at-rest) titkosít, merevlemezen és a biztonsági mentési adathordozón is. Használatban és szállításban lévő adatokat nem véd (data-in-use, data-in-transit).
\\- A módszer biztosítja, hogyha még el is lopják a fizikai adathordozót, akkor sem férnek hozzá a tolvajok a rajta lévő adatokhoz.
\\- Mivel az összes adatot titkosítja, ezért nem szükséges speciális módon rendezni az adatokat.
\\- Adatok titkosítási a tároláskor történik, visszafejtésük pedig rendszer memóriába való hívásakor történik.
\\- Szimmetrikus kulcsot használ a kódoláshoz.
\\- „A vállalatok jellemzően a TDE-t alkalmazzák az olyan megfelelési problémák megoldására, mint például a PCI DSS, amely megköveteli a nyugalmi adatok védelmét.”
\\- Microsoft, Oracle, IBM is alkalmazza ezt a módszert az adatbázis fájlok titkosítása érdekében.

\vspace{25pt}
\noindent\textbf{Column Level Encryption (Oszlop szintű titkosítás):}\newline
\noindent - Relációs adatmodell esetén egy adatbázis táblákból, oszlopokból, sorokból és cellákból vagy mezőkből áll. Ahogy a nevéből is következik, ez a módszer egy ilyen adatbázis egy sorát titkoítja.
\\- Független oszlopokat lehet titkosítani. Az oszlop összes adatát titkosítja kivétel nélkül. 
\\- Előnye, hogy könnyedén megkülönböztethető az érzékeny és a nem érzékeny adat, illetve külön kulcs használható minden oszlop titkosításához, így növelve a biztonságot. Sokkal rugalmasabb, mint a teljes adatbázist titkosító TDE.
\\- Hátrány az előnyeiből fakad. Több oszlop több kulccsal való titkosítása az adatbázis teljesítményének csökkenéséhez vezethet. Lassabban lehet keresni és indexelni is.
\\- Akkor használatos, ha nincs szükség teljes adatbázis titkosításra, hanem megkülönböztethető, hogy mely oszlopok tárolnak érzékeny adatot és melyek nem.
\\-Microsoft, Oracle, IBM , MyDiamo és még sok más cég használja ezt a titkosítási módszert.

\vspace{25pt}
\noindent\textbf{Field / Cell Level Encryption (Cella szintű titkosítás):}\newline
\noindent- Relációs adatmodell esetén használható.
\\- Kiválasztható, hogy pontosan melyik mezőt szeretnénk titkosítani. 
\\- Előnyei és hátrányai megegyeznek az előbb ismertetett Column Level Encryption-nel. 
\\- Nincs minden esetben szükség a mezők dekódolására, lehetőség van egyenlőség vizsgálatra.
\\- Akkor használatos, ha nincs szükség teljes adatbázis titkosításra, hanem megkülönböztethető, hogy mely cellák tárolnak érzékeny adatot és melyek nem.
\\- Microsoft, Oracle, IBM , MyDiamo és még sok más cég használja ezt a titkosítási módszert.




\subsection{Egyéb titkosítási módszerek}

\vspace{25pt}
\noindent\textbf{Filesystem Encryption:} \newline
\noindent- Fájlrendszer titkosítás, szokás még fájl / mappa titkosításnak, FBE-nek (file-based encryption) is nevezni.
\\- Célja adott fájl tartalmának titkosítása
\\- Előnye, hogy minden fájlt külön kulccsal lehet titkosítani, így növelve a biztonságot.
\\- A kriptográfiai kulcs addig van a memóriában, amíg az adott fájl meg van nyitva.
\\- Aki fizikailag hozzáfér a tároló számítógéphez, az láthatja, hogy milyen nevű fájlok találhatóak a rendszeren, holott a tartalmukat nem tudja megnézni, amíg nem ismeri a kulcsot.
\\- Olyan adatokat is képes titkosítani, amelyek nem részei az adatbázis rendszernek.
\\- Csökkenti a teljesítményt és operációs rendszer hozzáférést is kíván a használatához.
\\- Teljesítményproblémák miatt nem igazán alkalmazzák, de ennek elenére kis felhasználószámú rendszerek esetében ajánlják.


\vspace{25pt}
\noindent\textbf{Full Disk Encryption:} \newline
\noindent- Teljes merevlemez titkosítás
\\- Az egész merevlemez tartalma titkosításra kerül. 
\\- Általában ugyanazt a kulcsot használja az egész meghajtó titkosításához, ezért futásidőben az összes adat visszafejthető. Néhány módszer több kulcsot is használ különböző ’fejezetek’(?) titkosításához.
\\- Nagy hátránya, ha a támadó futásidőben fér hozzá a számítógéphez, minden fájl elérhető számára. 

\vspace{25pt}
\noindent\textbf{Application Level Encryption:} \newline
\noindent- Kódolás és dekódolás adatátvitel és tárolás előtt történik.
\\- Maga az alkalmazás végzi a titkosítási folyamatot.
\\- Az adat csak a megfelelő alklamazáson keresztül érhető el. Egy hacker-nek szüksége van az adatbázis és az adatokat használó alkalmazásra is az adatok visszafejtéséhez.
\\- Negatívuma lehet, ha ezt a titkosítási módszert szeretné alkalmazni egy cég, akkor maguknak kell implementálniuk, ami egy nem informatikai cég esetében nehéz lehet.
\\- Másik hátránya, a kulcsok kezelésének az összetettsége is megnövekedhet, ha több különböző alkalmazásnak kell egy adatbázishoz hozzáférnie, írnia, olvasnia.



\newpage \Section{Szinkronizáció}

Szóba került egy probléma többgépes alkalmazás készítése esetében, mégpedig az adatok szinkronizálása különböző eszközökön.
\vspace{5pt} \\ Az adatszinkronizálás egy olyan folyamat, ami alatt a különböző eszközökön eltárolt információkat megpróbáljuk össze egyeztetni, összhangba hozni.
\vspace{5pt} \\Szükséges megoldani ezt a problémát alkalmazások készítésekor, hiszen nem szeretnénk ugyanazt az információt bevinni a rendszerünkbe, amit már egy másik eszközön egyszer megtettünk. Legtöbb esetben szeretnénk azonos adatokat elérni minden készüléken.
\vspace{5pt} \\Programozási szempontból ez adatbázisok és/vagy fájlok szinkronizálását jelenti.
\vspace{15pt} \\ A probléma részletes felvezetése a következőképpen nézne ki:
\vspace{5pt} \\ \indent Tegyük fel van két fájl, A és B, amelyek két lassú kommunikációs kapcsolattal (slowcommunication link) összekötött gépen vannak. A-t úgy szeretnénk frissíteni, hogy tartalma megegyezzen B tartalmával.  Ha A nagyméretű, akkor A másolása B-re lassú lesz. Gyorsabbá tétel érdekében A-t tömöríthetjük küldés előtt, de ez nem egy perfekt megoldás.
\vspace{5pt} \\ Tegyük fel, hogy A és B nagyon hasonló, mondjuk ugyanazon eredeti fájlból származnak. Felgyorsítás érdekében a hasonlóság valamilyen módon történő kihasználása lenne a logikus lépés. Gyakori módszer, hogy A és B közötti különbségeket küldik el egymásnak, majd ebből a listából a fájl rekonstruálásra kerül. 
\vspace{5pt} \\ A probléma az ilyen módszerekkel, hogy a különbségek kialakításához olvashatónak kell lennie mindkét fájlnak, ami csak úgy érhető el, ha a kapcsolat egyik végén mindkét fájl rendelkezésre áll. Ha nem található meg egy eszközön mindkét fájl, akkor nem használhatók ezek a módszerek.
\vspace{5pt} \\ Mondhatjuk, hogy ez a probléma felvetés lokális adatbázis fájlok szinkronizálása esetén is fennáll.
\vspace{15pt} \\ A szinkronizáció megoldása nem egy triviális probléma, nem is találtam rá specifikus 'how-to' jellegű leírást, lépésekre szedett protokollt, ahogy két fájl szinkronizálási folyamata kéne kinézzen.
\\ Megtaláltam viszont a szinkronizációnak két fajtáját, amik a következők:

\subsection{Egyirányú szinkronizáció (One-way synchronization)}
Szokás még fájltükrözésnek (file mirroring), fájlreplikációnak (file replication) és fájlmentésnek (file backup) nevezni.
\vspace{5pt} \\A fájlok várhatóan csak egy helyen változnak. A változtatások egyeztetése érdekében a szinkronizálási folyamat egy irányba másolja a fájlokat. A két tárolási helyszín nem tekinthető egyenértékűnek. Az egyik helyszín a forrás (source), a másik pedig a cél (target). Bármilyen változtatás a forrásba tükröződni fog a célba. A célon elvégzett változtatások nem fognak a forráson replikálódni.Ha ez a folyamat végigmegy, azt lehet mondani, hogy a forrás tükrözve van a célba.
\vspace{5pt} \\Ez a módszer a forrás pontos másolatát hozza létre a célba. Hasznos és hatékony biztonsági mentés szempontjából, mivel csak a változtatott / új fájlok másolódnak.


\subsection{Kétirányú szinkronizáció (Two-way synchronization)}
Gyors szinkronizálásnak is szokás nevezni (fast sync).
\vspace{5pt} \\Ez a folyamat mindkét irányba másolja a fájlokat. A fájlok várhatóan mindkét helyen változnak, a két (vagy több) hely egyenértékűnek tekinthető.
\vspace{5pt} \\Célja, hogy két vagy több hely azonos legyen egymással.



\Section{Kulcskezelés - Key management}
Kulcs előállítását, cseréjét, tárolását és használatát jelenti.
\vspace{5pt} \\Komplexitás szempontjából, minél több alkalmazás adata kerül titkosításra, úgy nő általában a tárolandó és kezelendő kulcsok száma is.
\vspace{5pt} \\Kulcsok nem megfelelő tárolása és kezelése adatszivárgást eredményezhet. Ha a kulcskezelő rendszer valamilyen oknál fogva elveszti vagy törli a kulcsokat, akkor a titkosított adatok is elvesznek, feltéve ha nem készült a kulcsokról másolat.
\vspace{5pt} \\Vitatható, de úgy gondolom, hogy a megfelelő kulcskezelés a titkosítási folyamat legfontosabb része.
\vspace{5pt} \\Kimondottan a kulcskezelésre számos úgynevezett kulcskezelő rendszert (KMS) létrehoztak. Egy ilyen rendszer magába foglalja a kulcsok biztonságos generálását, cseréjét, tárolását.
\vspace{5pt} \\Kulcskezeléshez tartozó fogalom a ’certificate’ azaz tanúsítvány. A certificate egy szabványosított módszer egy adott felhasználó / applikáció / szerver hitelességének igazolására.
\vspace{5pt} \\Általában a maximum biztonság elérése érdekében egy harmadik-oldali ún. ’certificate authority’(CA) azaz tanúsítvány kiadó felel a certificate-ek kiosztásáért. Egy certificate a következő információkat tartalmazhatja:
\vspace{5pt} \\-Szervezeti információk: egyértelmű vállalat / szervezet azonosítók pl név vagy cím.
\vspace{5pt} \\-Certificate authority neve: a certificate előállítója ezzel az információval azonosítja magát.
\vspace{5pt} \\-Digitális aláírás: az előállító ezzel az aláírással látja el a certificate-et, hogy ellenőrizze annak hitelességét. A megfelelő CA ellenőrzi a tanúsítványt, hogy egy hiteles szolgáltatótól származik-e.


























































































