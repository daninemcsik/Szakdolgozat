\Chapter{Koncepció}

\Section{Személyes Információszervezés}
A személyes információszervezés (PIM - Personal Information Management) olyan folyamat ami alatt egy ember megpróbálja az általa hasznosnak vélt információkat összegyűjteni, rendszerezni, tárolni \cite{enwiki:1076787930}. A folyamat lehet fizikai vagy digitális.
\vspace{5pt}\\ Fontos, személyes információ lehet például:
\begin{itemize}
	\item Személy hivatalos adatai, okmányai
	\item Telefonszámok, e-mail címek, elérési adatok
	\item Képek, videók, TV adások
	\item Időpontok, tapasztalatok, emlékek
	\item Költségek, kiadások
	\item Zenék 
	\item Érdeklődési körökkel, hobbikkal kapcsolatos információk
\end{itemize}
Ez a lista személyről személyre változik, minden ember különbözik, más információkat talál fontosnak.
\vspace{5pt}\\ Nagyon előnyös, ha valamilyen formában megvalósítjuk a személyes információszervezést az életünkben. 
\\Gondoljunk bele, hányszor volt már olyan, hogy egy időpontot elfelejtettünk, nem találtunk meg egy weboldalt, nem emlékeztünk egy születésnapra vagy egy leadási határidőre. Egy PIM rendszer lényege ilyen információknak a struktúrált, összeszedett tárolása, hogy megkönnyítse az adott személy mindennapjait, hiszen ha egy helyen tárolunk minden fontos információt, akkor a jövőben tudni fogjuk, hogy hol keressük őket.

\newpage \Section{Adattárolás}
Az adatok tárolása megoldható adatbázisok segítségével és fájlokba.

\subsection{XML és JSON}
Adatátviteli formátumok közül a leggyakrabban használtak közé tartozik az XML és a JSON \cite{nurseitov2009comparison}. Céljuk, hogy platform-független, programozási nyelv független, alkalmazás független adatleíró formátumot adjanak, ami képes adatok nagy mértékű átvitelére és feldolgozására. A két formátum részletes összehasonlítását tartalmazza a \ref{tab:jsonandxml}-es táblázat.
\newline
\\ \underline{XML – eXtensible Markup Language}
\vspace{10pt}
\newline \noindent Rugalmas, könnyen értelmezhető emberek és számítógép számára is.  
\vspace{5pt}\\HTML-hez hasonló szintaxisú nyelv. Felépítése tag-ekből áll. Nem előre definiáltak a tag-ek, ez azt jelenti, hogy mi hozzuk létre őket. Egy példa tag felépítése: <autó>…. </autó>.
\vspace{5pt}\\Első a nyitó tag, a második, ferde vonallal jelölt tag a záró tag.
\\Tag-ek között megadhatjuk a tartalmukat (pl. autó márka), vagy további tag-ekkel, amiket gyerekelemek (pl rendszám, szín) nevezünk, tovább konkretizálhatjuk a tartalmat.
\\Üres tag megadására is lehetséges, aminek nincs tartalma (pl.:<autó/>).
\vspace{5pt}\\Egy XML dokumentum robosztus méreteket is elérhet a tag-ekből álló felépítése miatt.
\newline

\noindent \underline{JSON – JavaScript Object Notation}
\vspace{10pt}
\newline \noindent Rugalmas, könnyen értelmezhető emberek és számítógép számára is
\vspace{5pt}\\ Adatok név-érték párokban szerepelnek (pl. ”keresztnév” : ”Dániel”).
\\Több egymást követő adat vesszővel van elválasztva egymástól (pl.: ”márka” : ”Mercedes”, ”szín” : ”piros”).
\\ \{ \} – objektumot tartalmaz (pl. \{”keresztnév” : ”Dániel”, ”vezetéknév” : ”Nemcsik”\})
\newline [ ] –tömböt tartalmaz (pl. \{ ”diákok”:[
\\  \indent	\{”keresztnév” : ”Dániel”, ”vezetéknév” : ”Nemcsik”\},
\\  \indent \{”keresztnév” : ”Tibor”, ”vezetéknév” : ”Nagy”\}
]\}

\begin{table}[H]
	\centering
	\caption{JSON és XML összehasonlító táblázat}
	\label{tab:jsonandxml}
	\medskip
	\begin{tabular}{|p{7.2cm}|p{7.2cm}|}
		\hline
		\textbf{JSON} & \textbf{XML} \\
		\hline
		JavaScript nyelven alapszik & SGML szabványból származtatott \\
		\hline
		Tábla(kulcs-érték) felépítés & Fa felépítés \\
		\hline
		Könnyedén értelmezhető & Tag felépítés miatt, aki nem tudja hogy kell értelmezni, annak nehezebb dolga lesz mint egy JSON file-lal lenne\\
		\hline
		Nem támogat kommenteket & Támogat kommenteket \\
		\hline
		Adatok szerver és böngésző oldal közötti hordozására preferált & Szerver oldali információ tárolásra preferált\\
		\hline
		Kevésbé terjedelmes és gyorsabb & Több szó szerepel benne, így nagyobb terjedelmű. Lassabb.\\
		\hline
		Kisebb file méret & Nagyobb file méret\\
		\hline
		String, számok, tömbök, objektumok és boolean értékeket támogat & Komplexebb adattípusokat is támogat például képek, nem-primitív adattípusok, stb..\\
		\hline
		Minden böngésző támogatja & Legtöbb böngésző támogatja\\
		\hline
		Adatcsere formátumtípusú & Jelölőnyelv formátumtípusú\\
		\hline
		Adat megjelenítésére nem alkalmas & Mivel jelölőnyelv (markup language), ezért adat megjelenítésére is alkalmas\\
		\hline
		Egy szabványos javascript függvénnyel használat előtt elemezni/felbontani/szeparálni (parse) kell a file-t. & Adott programozási nyelvnek megfelelően kell elemezni/felbontani/szeparálni (parse) a file-t.\\
		\hline
		Könnyedén elemezhető, kevés kód szükséges hozzá & Nehezebben elemezhető a tartalma\\
		\hline
		Adat-orientáltnak nevezik & Dokumentum-orientáltnak nevezik\\
		\hline
	\end{tabular}
\end{table}

\vspace{6pt}
\noindent Tegyük fel, hogy egy alkalmazás adatai vagy JSON vagy XML típusú fájlként, lokálisan tároljuk a felhasználó eszközén. \textbf{Lokális fájl} alapú tárolást az alábbi tulajdonságokkal jellemezhetjük:
\begin{itemize}
	\item Gyors és egyszerű hozzáférés az adatokhoz.
	\item Könnyedén implementálható megoldás, évtizedek óta alkalmazott módszer az adatok lokális fájlként való tárolása.
	\item Könnyedén átmásolható fizikai adathordozóra, így potenciálisan növelve a védelmet.
	\item Adathordozó elhagyása / lopása, fájlok véletlen törlése biztonsági mentés nélkül adat teljes elvesztését eredményezheti.
	\item Több hasonló fájl esetén gyakori jelenség az adatduplikáció. 
\end{itemize}


\newpage
\subsection{Adatbázis}

Adatbázisok kapcsán minden bizonnyal hallottunk már az SQL-ről, legtöbb relációs adatbázis ezt a nyelvet használja, egy struktúrált lekérdező nyelv. Feltételezem az SQL ismeretét, ezért bővebb jellemzésére nem tér ki a dokumentum. 
\\Adatbázisok viszont lehetnek különböző típusúak, különböző adatmodellűek. Két típust fogunk megnézni, a relációs adatbázist és NoSQL adatbázist. Ennek a két adatmodellnek egy rövid összehasonlítását tartalmazza a \ref{tab:rdbmsandnosql}-es táblázat.

\begin{table}[H]
	\centering
	\caption{Relációs és NoSQL adatbázis összehasonlító táblázat}
	\label{tab:rdbmsandnosql}
	\medskip	
	\begin{tabular}{|p{7.2cm}|p{7.2cm}|}
		\hline
		\textbf{Relációs adatbázis} & \textbf{NoSQL adatbázis} \\
		\hline
		Adatok összefüggése táblák közötti kapcsolattal határozható meg. &  Dinamikus séma, nem struktúrált adatnak nevezzük.\\
		Struktúrált formátumú tárolási mód. & Nem relációs adatmodell.\\
		& Adatok gyakran kulcs-érték párokban szerepelnek vagy JSON formátumban.\\
		\hline
		Felépítéséből adódóan adatmodell változásokat nehezebb implementálni. & Könnyebb implementálni adatmodell változásokat, mivel az adatok nem biztos, hogy kapcsolódnak egymáshoz. \\
		\hline
		Fix, meghatározott adat felépítés esetén a legjobb használni. & Nem struktúrált, komplex adatok esetén gyorsabb, kisebb erőforrás igényű lehet, mint egy relációs adatbázis használata.\\
		\hline
	\end{tabular}
\end{table}

\vspace{10pt}
\noindent Azonfelül, hogy egy adatbázis milyen adatmodellű, lehet lokális, a felhasználó eszközén tárolt és távoli, felhőalapú, cloud adatbázis.
\\ Térjünk ki a lokális és felhőalapú adatbázisok közötti összehasonlításra: \newline


\noindent \textbf{Lokális adatbázis} \cite{enwiki:1085248519}, hasonló \underline{előnyei} vannak, mint egy helyileg tárolt fájlnak:
\begin{itemize}
	\item Egyszerű hozzáférés az adatokhoz, teljes adat kontrollálás.
	\item Nincs szükség internetkapcsolatra.
	\item Gyorsabb, mint egy távoli adatbázis, mivel gyorsabb a helyi lemez elérése, mint egy távoli esetében a hálózaton keresztüli kommunikáció.
\end{itemize}
\noindent \underline{Hátrány:}
\begin{itemize}
	\item Nehéz az adatmegosztás külső résztvevővel (másik számítógép lokális adatbázisa).
	\item Ha az eszköz olyan állapotba kerül, hogy a fájlokhoz nem lehet hozzáférni, akkor elveszlik minden adat.
\end{itemize}

\newpage \noindent \textbf{Felhőalapú adatbázis} \cite{chandra2012study}:
\begin{itemize}
	\item Bárhonnan hozzáférhető, internet elérés szükséges. Negatívuma ugyanebből származik, ha nincs internet kapcsolat, nem használható.
	\item Adatok nem helyileg a számítógépen tárolódnak, emiatt lassabb lehet a hozzáférés. Figyelembe kell venni, hogy más szerverek / alkalmazások is használhatják ugyanazt a hálózatot, ami szintén befolyásolhatja az elérési sebességet.
	\item Teknikai hibák ellenére (felhasználó eszköze meghibásodik) az információ sértetlen marad a távoli adatbázison.
	\item Cloud adatbázis használata esetén a szinkronizáció bonyolult problémája kikerülhető. Több eszköz használhatja ugyanazt az adatbázist, ugyanazokkal az adatokkal dolgozhatnak.
\end{itemize}

\newpage
\Section{Adat állapot}
Adat állapota alatt három lehetséges esetet értünk, data-at-rest, data-in-motion és data-in-use. Az alfejezet a data-at-rest és data-in-motion \cite{varriale2016secube} adatállapot fogalmát, biztonságukkal kapcsolatos információkat írja le.

\subsection{Data-at-rest}
Nyugvó adatnak olyan adatokat nevezünk, amelyek nem mozognak eszközről eszközre, vagy hálózatról hálózatra. Általában merevlemezen vagy pendrive-on tárolódnak.
\vspace{5pt}\\A nyugalmi adatvédelem célja a bármilyen eszközön vagy hálózaton tárolt inaktív adat védelme.
\vspace{5pt}\\Nyugvó adatokat általában kevésbé sebezhetőnek tartják, mint a mozgásban lévő adatokat (data-in-motion), gyakran értékesebbnek is tekinthető tartalmuk. Nyugvó adatok esetében az ellopható információ mennyisége sokkal nagyobb lehet mint az éppen úton levő adatoké (data-in-motion).
\vspace{5pt}\\A nyugvó adatok biztonsága a szükséges óvintézkedések megtételétől függ.
\\Egy cég legtöbb esetben adatit saját hálózatán belül tárolja, ettől függetlenül veszélyben lehetnek rosszindulatú külső és belső fenyegetésekkel szemben. Egy betolakodó könnyedén elérheti az adott cég adatait, ha sikerül jogtalanul hozzáférnie egy számítógépükhöz vagy egy lopott eszközt feltörnie.
\vspace{5pt}\\Data-at-rest típusú adatok védelme érdekében az egyik legjobb, legegyszerűbb és leggyakrabban alkalmazott módszer ezek titkosítása.


\subsection{Data-in-motion / data-in-transit}
Mozgásban lévő adatnak vagy tranzitadatnak olyan adatokat nevezünk, amelyek aktívan mozgásban vannak egyik helyről a másikra, vagy az interneten vagy magánhálózaton keresztül.
\vspace{5pt}\\Mivel az adatok mozgásban vannak, ezért kevésbé biztonságosnak tekinthetők. Célja olyan adatok védelme amelyek például belső hálózaton belül mozognak, vagy helyi tárolóeszközről felhőtípusú tárolóeszközre.
\vspace{5pt}\\Tranzitadat esetében is egy kiváló biztonsági intézkedés az adatok titkosítása. Védi az adatokat, ha két fél közötti kommunikációt ’lehallgatják’.
Ez a védelem az adatok titkosításával biztosítható, még mozgatás előtt, vagy magának a kommunikációs csatornának titkosításával. Ez lehet talán a legfontosabb része a mozgásban lévő adatok védelmének, illetve a megfelelő kulcskezelés. Végpontok hitelesítése és adat érkezésekor való visszafejtése és ellenőrzése is tovább fokozhatja a védelmet. 

\newpage
\Section{Adattitkosítási módszerek}

Ezt a szegmenst két részre bontható. Először az adatbázis titkosítási módszerek kerülnek bemutatásra, ezek olyan titkosítási eljárások, amik kizárólag adatbázisok és adataik titkosítására lettek kifejlesztve. Másik csoportba minden más olyan titkosítási módszer tartozik, amik nem adatbázishoz köthetők (fájlok, fájlrendszerek).

\subsection{Adatbázis titkosítási módszerek} 

Olyan folyamatot nevezünk adatbázis titkosításnak, ami egy vagy több titkosítási algoritmus használatával az adatbázisban tárolt adatokat titkosított szöveggé (cipher text) alakítja. Ez a szöveg értelmezhetetlen a megfelelő kulcs ismerete nélkül \cite{bouganim2009database}.
\\Célja, hogy megvédje az adatainkat a potenciális fenyegetésektől. Ha egy hacker valahogy sikeresen feltöri az adatbázist, akkor számára értéktelen, értelmezhetetlen szöveggel fog találkozni.
\\Többféle titkosítási technika is létezik, melyek közül a legleterjedtebbek következnek.\newline

\noindent\textbf{Transparent Data Encryption (TDE) (Átlátható adat titkosítás):}\newline
\begin{itemize}
	\item Teljes adatbázist, nyugvó adatokat (data-at-rest) titkosít merevlemezen és a biztonsági mentési adathordozón is. Használatban és szállításban lévő adatokat nem véd (data-in-use, data-in-transit).
	\item A módszer biztosítja, ha még el is lopják a fizikai adathordozót, akkor sem férnek hozzá a tolvajok az eszközön lévő adatokhoz.
	\item Mivel az összes adatot titkosítja, ezért nem szükséges speciális módon rendezni az adatokat.
	\item Adatok titkosítása tároláskor történik, visszafejtésük pedig rendszer memóriába való hívásakor történik.
	\item Szimmetrikus kulcsot használ a kódoláshoz.
	\item Microsoft, Oracle, IBM is alkalmazza ezt a módszert adatbázis fájlok titkosítása érdekében.
\end{itemize}


\vspace{25pt}
\noindent\textbf{Column Level Encryption (Oszlop szintű titkosítás):}\newline
\begin{itemize}
	\item Relációs adatmodell esetén egy adatbázis táblákból, oszlopokból, sorokból és cellákból/mezőkből áll. Ahogy a módszer nevéből is következik, relációs adatbázis egy sorát titkoítja.
	\item Lehetséges független oszlopok titkosítása. Egy oszlop összes adatát titkosítja kivétel nélkül. 
	\item Akkor használatos, ha nincs szükség teljes adatbázis titkosításra, egyértelműen megkülönböztethető, hogy mely oszlopok tárolnak érzékeny adatot és melyek nem.
	\item Előnye, hogy könnyedén megkülönböztethető az érzékeny és a kevésbé érzékeny adat, illetve külön kulcs használható minden oszlop titkosításához, így növelve a biztonságot. Sokkal rugalmasabb, mint a teljes adatbázist titkosító TDE.
	\item Hátránya az előnyeiből fakad. Több oszlop több kulccsal való titkosítása az adatbázis teljesítményének csökkenéséhez vezethet. Lassabban lehet keresni és indexelni is.
	\item Microsoft, Oracle, IBM , MyDiamo és még sok más cég használja ezt a titkosítási módszert.
\end{itemize}

\vspace{25pt}
\noindent\textbf{Field / Cell Level Encryption (Cella/mező szintű titkosítás):}\newline
\begin{itemize}
	\item Relációs adatmodell esetén használható.
	\item Kiválasztható, hogy pontosan melyik mezőt szeretnénk titkosítani. 
	\item Akkor használatos, ha nincs szükség teljes adatbázis titkosításra, hanem megkülönböztethető, hogy mely cellák tárolnak érzékeny adatot és melyek nem.
	\item Előnyei és hátrányai megegyeznek a Column Level Encryption-ével. 
	\item Nem biztos, hogy minden esetben szükséges a mezők dekódolása, lehetőség van egyenlőség vizsgálatra.
	\item Microsoft, Oracle, IBM , MyDiamo és még sok más cég használja ezt a titkosítási módszert.
\end{itemize}

\subsection{Egyéb titkosítási módszerek}

\vspace{15pt}
\noindent\textbf{Filesystem Encryption:} \cite{enwiki:1020951139} \newline
\begin{itemize}
	\item Fájlrendszer titkosítás, szokás még fájl / mappa titkosításnak, FBE-nek (file-based encryption) is nevezni.
	\item Célja fájl(ok) tartalmának titkosítása.
	\item Előnye, hogy minden fájlt külön kulccsal titkosítható, így növelve a biztonságot.
	\item Egy kriptográfiai kulcs addig van a memóriában, amíg az adott fájl meg van nyitva.
	\item Aki fizikailag hozzáfér a tároló számítógéphez, az láthatja, hogy milyen nevű fájlok találhatóak a rendszeren, holott a tartalmukat nem tudja megnézni, amíg nem ismeri a kulcsot.
	\item Olyan adatokat is képes titkosítani, amelyek nem részei egy adatbázis rendszernek.
	\item Csökkenti a teljesítményt és operációs rendszer hozzáférést is kíván a használatához.
	\item Teljesítmény problémák miatt nem igazán alkalmazzák, de ennek ellenére kis felhasználószámú rendszerek esetében ajánlott.
\end{itemize}


\vspace{25pt}
\noindent\textbf{Full Disk Encryption:} \cite{enwiki:1084723482} \newline
\begin{itemize}
	\item Merevlemez teljes tartalma titkosításra kerül. 
	\item Általában ugyanazt a kulcsot használja az egész meghajtó titkosításához, ezért futásidőben az összes adat visszafejthető.
	\item Nagy hátránya, ha a támadó futásidőben fér hozzá a számítógéphez, minden fájl elérhető számára. 
\end{itemize}

\vspace{25pt}
\noindent\textbf{Application Level Encryption:} \cite{saxena2015application} \newline
\begin{itemize}
	\item Kódolás és dekódolás adatátvitel és tárolás előtt történik.
	\item Maga az alkalmazás végzi a titkosítási folyamatot.
	\item Az adat csak a megfelelő alklamazáson keresztül érhető el. Egy hacker-nek szüksége van az adatbázis és az adatokat használó alkalmazásra is az adatok visszafejtéséhez.
	\item Negatívuma lehet, ha ezt a titkosítási módszert szeretné alkalmazni egy cég, akkor maguknak kell implementálniuk, ami egy nem informatikai cég esetében bonyolult probléma lehet.
	\item Másik hátránya, a kulcsok kezelésének az összetettsége is megnövekedhet, ha több, különböző alkalmazásnak kell ugyanazon adatbázishoz hozzáférnie, írnia, olvasnia.
\end{itemize}


\newpage \Section{Szinkronizáció}

Szóba került egy probléma többgépes alkalmazás készítése esetében, mégpedig az adatok szinkronizálása különböző eszközökön \cite{nakatani2006data}.
\vspace{5pt} \\ Az adatszinkronizálás egy olyan folyamat, ami alatt a különböző eszközökön eltárolt információkat megpróbáljuk össze egyeztetni, összhangba hozni.
\vspace{5pt} \\Alkalmazások készítésekor szükséges a szinkronizáció problémájának megoldása, hiszen nem szeretnénk ugyanazt az információt bevinni a rendszerünkbe, amit már egy másik eszközön korábban megtettünk. Legtöbb esetben szeretnénk azonos adatokat elérni minden készüléken.
\vspace{5pt} \\Programozási szempontból ez adatbázisok és/vagy fájlok szinkronizálását jelenti.
\vspace{15pt} \\ A probléma részletes felvezetése a következőképpen nézne ki:
\vspace{5pt} \\  Tegyük fel van két fájl, A és B, amelyek két lassú kommunikációs kapcsolattal (slowcommunication link) összekötött gépen vannak. A-t úgy szeretnénk frissíteni, hogy tartalma megegyezzen B tartalmával.  Ha A nagyméretű, akkor A másolása B-re lassú lesz. Gyorsaság érdekében A-t tömöríthetjük küldés előtt, de ez nem egy perfekt megoldás.
\vspace{5pt} \\ Tegyük fel, hogy A és B nagyon hasonló, mondjuk ugyanazon eredeti fájlból származnak. Felgyorsítás érdekében a hasonlóság valamilyen módon történő kihasználása lenne a logikus lépés. Gyakori módszer, hogy A és B közötti különbségeket küldik el a felek egymásnak, majd ebből a listából a fájl rekonstruálásra kerül. 
\vspace{5pt} \\ A probléma az ilyen módszerekkel, hogy a különbségek kialakításához olvashatónak kell lennie mindkét fájlnak, ami csak úgy érhető el, ha a kapcsolat egyik végén mindkét fájl rendelkezésre áll. Ha nem található meg egy eszközön mindkét fájl, akkor nem használhatók ezek a módszerek.
\vspace{5pt} \\ Mondhatjuk, hogy ez a probléma felvetés lokális adatbázis fájlok szinkronizálása esetén is fennáll.
\vspace{15pt} \\ A szinkronizáció megoldása nem egy triviális probléma, nem is találtam rá specifikus 'how-to' jellegű leírást, lépésekre lebontott protokollt, ahogy két fájl szinkronizálási folyamata kéne történjen.
\\ Megtaláltam viszont a szinkronizációnak két fajtáját \cite{balaji2020blog}:

\subsection{Egyirányú szinkronizáció (One-way synchronization)}
Szokás még fájltükrözésnek (file mirroring), fájlreplikációnak (file replication) és fájlmentésnek (file backup) nevezni.
\vspace{5pt} \\A fájlok várhatóan csak egy helyen változnak. A változtatások egyeztetése érdekében a szinkronizálási folyamat egy irányba másolja a fájlokat. A két tárolási helyszín nem tekinthető egyenértékűnek. Az egyik helyszín a forrás (source), a másik pedig a cél (target). Bármilyen változtatás a forrásba tükröződni fog a célba. A célon elvégzett változtatások nem fognak a forráson replikálódni.Ha ez a folyamat végigmegy, azt lehet mondani, hogy a forrás tükrözve van a célba.
\vspace{5pt} \\Ez a módszer a forrás pontos másolatát hozza létre a célba. Hasznos és hatékony biztonsági mentés szempontjából, mivel csak a változtatott / új fájlok másolódnak.


\subsection{Kétirányú szinkronizáció (Two-way synchronization)}
Gyors szinkronizálásnak is szokás nevezni (fast sync).
\vspace{5pt} \\Ez a folyamat mindkét irányba másolja a fájlokat. A fájlok várhatóan mindkét helyen változnak, a két (vagy több) hely egyenértékűnek tekinthető.
\vspace{5pt} \\Célja, hogy két vagy több hely azonos legyen egymással.



\Section{Kulcskezelés - Key management}
Kulcs előállítását, cseréjét, tárolását és használatát jelenti \cite{rafaeli2003survey}.
\vspace{5pt} \\Komplexitás szempontjából, minél több alkalmazás adata kerül titkosításra, úgy nő általában a tárolandó és kezelendő kulcsok száma is.
\vspace{5pt} \\Kulcsok nem megfelelő tárolása és kezelése adatszivárgást eredményezhet. Ha a kulcskezelő rendszer valamilyen oknál fogva elveszti vagy törli a kulcsokat, akkor a titkosított adatok is elvesznek, feltéve ha nem készült a kulcsokról biztonsági másolat.
\vspace{5pt} \\Vitatható, de úgy gondolom, hogy a megfelelő kulcskezelés a titkosítási folyamat legfontosabb része.
\vspace{5pt} \\Kimondottan kulcskezelésre számos, úgynevezett kulcskezelő rendszereket (KMS) hoztak létre. Egy ilyen rendszer magába foglalja a kulcsok biztonságos generálását, cseréjét, tárolását.
\vspace{5pt} \\Kulcskezeléshez tartozó fogalom a ’certificate’ azaz tanúsítvány. A certificate egy szabványosított módszer egy adott felhasználó / applikáció / szerver hitelességének igazolására.
\vspace{5pt} \\A maximum biztonság elérése érdekében általában egy harmadik-oldali ún. ’certificate authority’(CA) azaz tanúsítvány kiadó felel a certificate-ek kiosztásáért. Egy certificate a következő információkat tartalmazhatja:
\begin{itemize}
	\item Szervezeti információk: egyértelmű vállalat / szervezet azonosítók pl név vagy cím.
	\item Certificate authority neve: a certificate előállítója ezzel az információval azonosítja magát.
	\item Digitális aláírás: az előállító ezzel az aláírással látja el a certificate-et, hogy ellenőrizhető legyen annak hitelessége. A megfelelő CA ellenőrzi a tanúsítványt, hogy egy hiteles szolgáltatótól származik-e.
\end{itemize}

