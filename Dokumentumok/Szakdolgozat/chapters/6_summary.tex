\Chapter{Összefoglalás}

Úgy gondolom, hogy sikerült a dolgozat elméleti részének egész jól lefednie a témát és a vizsgált problémákat.

Biztos vagyok benne, hogy sikerült kiválasztanom a megfelelő titkosítási algoritmust és módszert mind a tesztek és az alkalmazás szempontjából is, hiszen az AES egy nemzetközileg elismert és tesztelt, erős és biztonságos titkosítási algoritmus.

Az alkalmazás végső verziójával teljes mértékben meg vagyok elégedve. Sikerült kialakítanom szinte minden téren úgy az applikációt, ahogy azt témaválasztáskor elterveztem.

Úgy érzem, a szövegszerkesztőt még lehetne  tovább fejleszteni és tökéletesíteni, szinte majdnem az elképzeléseknek megfelelően működik, de úgy gondolom több időt kellett volna a fejlesztésére szánnom. \\Ugyanakkor észben kell tartani, hogy az alkalmazás elsődleges funkciója nem a szöveg szerkeszthetősége volt, hanem adatok jegyzet-szerű, struktúrált és biztonságos tárolása, valamint az elérhetőség, rendelkezésre állás, amiket szerintem sikerült jól kiviteleznem.

Összefoglalva meg vagyok elégedve a munkámmal és a dolgozattal, sok hasznos dolgot tanultam adatbiztonságtól kezdve idegen kód olvasásáig, amik a jövőben biztosan hasznomra válnak.

Ahogy az 5. fejezet végén írtam, az alkalmazást használni és fejlesztését folytatni fogom a jövőben.


