\Chapter{Bevezetés}

A dolgozat fő célja egy Java asztali alkalmazás elkészítése volt, ami képes többgépes környezetben is biztonságosan eltárolni a felhasználó adatait és visszaadni azokat szinkronizált módon az eszközök között.

A program készítése során különböző problémákat kellett megoldanom, mint például a megfelelő titkosítási algoritmus kiválasztása, adatszinkronizáció elérése különböző eszközök között, esztétikus Java grafikus felhasználói felület létrehozása.

A szakdolgozat a felmerülő problémák potenciális megoldási lehetőségeit és elméleti hátterét mutatja be.

Megpróbálja részletesen ismertetni az adattitkosítás során előfordulható problémákat.
 
Jellemzi, összeveti ezen problémák megoldási lehetőségeit. Ezek közé értendő az adatszerkezetek és adattárolási módok, adatbázis típusok, adatbázis titkosítási módszerek, titkosítási algoritmusok jellemzése.

% TODO: Bővíteni kellene legalább egy egész oldal terjedelemig!

